%% -!TEX root = AUTthesis.tex
% در این فایل، عنوان پایان‌نامه، مشخصات خود، متن تقدیمی‌، ستایش، سپاس‌گزاری و چکیده پایان‌نامه را به فارسی، وارد کنید.
% توجه داشته باشید که جدول حاوی مشخصات پروژه/پایان‌نامه/رساله و همچنین، مشخصات داخل آن، به طور خودکار، درج می‌شود.
%%%%%%%%%%%%%%%%%%%%%%%%%%%%%%%%%%%%
% دانشکده، آموزشکده و یا پژوهشکده  خود را وارد کنید
\faculty{دانشکده مهندسی برق}
% گرایش و گروه آموزشی خود را وارد کنید
\department{گرایش کنترل}
% عنوان پایان‌نامه را وارد کنید
\fatitle{بهبود عملکرد ربات اسپایدر و پیاده سازی الگوریتم برنامه ریزی مسیر برای آن
	\\[.75 cm]
	پایان‌نامه}
% نام استاد(ان) راهنما را وارد کنید
\firstsupervisor{دکتر محمداعظم خسروی}
%\secondsupervisor{استاد راهنمای دوم}
% نام استاد(دان) مشاور را وارد کنید. چنانچه استاد مشاور ندارید، دستور پایین را غیرفعال کنید.
%\firstadvisor{نام کامل استاد مشاور}
%\secondadvisor{استاد مشاور دوم}
% نام نویسنده را وارد کنید
\name{سجاد }
% نام خانوادگی نویسنده را وارد کنید
\surname{قدیری}
%%%%%%%%%%%%%%%%%%%%%%%%%%%%%%%%%%
\thesisdate{شهریور 1402}

% چکیده پایان‌نامه را وارد کنید
\fa-abstract{
	در این پایان‌نامه، به بررسی و طراحی یک ربات عنکبوتی چهار پا با استفاده از دوربین به عنوان سامانه‌ای برای تشخیص موانع با رنگ‌های مختلف و برنامه‌ریزی مسیر در محیط‌های پویا و متغیر می‌پردازیم. این سامانه از یک الگوریتم مبتنی بر داده استفاده می‌کند تا با تجزیه و تحلیل تصاویر از دوربین، به تشخیص موانع در مسیر خود بپردازد و برنامه‌ریزی مناسبی را برای جلوگیری از تصادفات و تصادمات انجام دهد.
}


% کلمات کلیدی پایان‌نامه را وارد کنید
\keywords{کلیدواژه اول، ...، کلیدواژه پنجم (نوشتن سه تا پنج واژه کلیدی ضروری است)}



\AUTtitle
%%%%%%%%%%%%%%%%%%%%%%%%%%%%%%%%%%
\vspace*{7cm}
\thispagestyle{empty}
\begin{center}
	\includegraphics[height=5cm,width=12cm]{besm}
\end{center}