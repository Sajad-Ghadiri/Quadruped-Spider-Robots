
%% -!TEX root = AUTthesis.tex
% در این فایل، عنوان پایان‌نامه، مشخصات خود، متن تقدیمی‌، ستایش، سپاس‌گزاری و چکیده پایان‌نامه را به فارسی، وارد کنید.
% توجه داشته باشید که جدول حاوی مشخصات پروژه/پایان‌نامه/رساله و همچنین، مشخصات داخل آن، به طور خودکار، درج می‌شود.
%%%%%%%%%%%%%%%%%%%%%%%%%%%%%%%%%%%%
% دانشکده، آموزشکده و یا پژوهشکده  خود را وارد کنید
\faculty{دانشکده برق}
% گرایش و گروه آموزشی خود را وارد کنید
\department{گرایش کنترل}
% عنوان پایان‌نامه را وارد کنید
\fatitle{
	بهبود عملکرد ربات اسپایدر و پیاده سازی الگوریتم برنامه ریزی مسیر برای آن}
% نام استاد(ان) راهنما را وارد کنید
\firstsupervisor{دکتر محمد اعظم خسروی}
%\secondsupervisor{پروفسور محمد باقر منهاج}
% نام استاد(دان) مشاور را وارد کنید. چنانچه استاد مشاور ندارید، دستور پایین را غیرفعال کنید.
%\firstadvisor{دکتر محمد باقر منهاج}
%\secondadvisor{استاد مشاور دوم}
% نام نویسنده را وارد کنید
\name{سجاد }
% نام خانوادگی نویسنده را وارد کنید
\surname{قدیری}
%%%%%%%%%%%%%%%%%%%%%%%%%%%%%%%%%%
\thesisdate{تابستان 1402}

% چکیده پایان‌نامه را وارد کنید
\fa-abstract{
رربات‌ها از جدیدترین دستاورد‌های انسان و تلفیقی از فناوری در حوزه‌‌های مختلفی همچون کنترل، هوش مصنوعی و اتوماسیون و... هستند. امروزه هوشمندسازی فناوری‌های پیشین موردتوجه بوده و ربات‌ها نیز از آن مستثنی نیستند. از میان انواع ربات‌ها آن دسته‌ای که در ساخت آنها از طبیعت الهام گرفته شده است دارای چالش‌های نوینی بوده‌اند. ربات‌های عنکبوتی چهارپا نمونه‌ای از این دسته هستند. این ربات‌ها از لحاظ مکانیزم حرکتی و همچنان کاربرد‌ دارای تفاوت‌هایی با ربات‌های چرخ‌دار و... هستند.
این پژوهش به تلاش‌هایی که در جهت بهبود عملکرد و مسیریابی ربات عنکبوتی چهارپا انجام شده می‌پردازد. بدین منظور قابلیت‌هایی به ربات اضافه گردیده و سپس به بهره‌گیری از این قابلیت‌ها در جهت بهبود عملکرد ربات تلاش شده است. در ابتدا یک ماژول چندجانبه که شامل دوربین است به ربات اضافه شده و به طراحی یک سرور باهدف تبادل اطلاعات به‌صورت بیسیم میان خود و ربات پرداخته می‌شود. از این ویژگی برای نظارت و ارزیابی صحت اطلاعات و همچنین تصمیم‌گیری‌های ربات در محیط استفاده خواهد شد. در ادامه تشخیص موانع موجود در محیط به کمک تشخیص رنگ و روش‌های پردازش تصویر با استفاده از زبان پایتون ارائه می‌گردد.
به‌منظور موقعیت‌سنجی ربات در محیط از کتابخانه‌ای که اساس کار آن بهره‌گیری از تگ‌های حاوی اطلاعات است، استفاده شده است.
همچنین برای هدف نهایی ربات یعنی طی‌کردن مسیری از نقطه شروع به نقطه پایان، یک الگوریتم مسیر‌یابی شبیه‌‌سازی شد. همچنین در جهت تلفیق این الگوریتم با ویژگی‌های ربات برای پیاده‌سازی عملی، تغییرات موردنیاز مطرح و بررسی می‌گردد.
در انتها آزمایش‌های عملی بر روی ربات با دو سناریو مختلف صورت‌گرفته و نتایج آن ارائه می‌گردد. ارائه چالش‌ها و پیشنهاداتی برای کارهای پیش ‌رو نیز مختصراً انجام گردیده است.		
}


% کلمات کلیدی پایان‌نامه را وارد کنید
\keywords{
	ربات عنکبوتی چهارپا ، دوربین، تشخیص موانع، مسیریابی با روش
	\lr{RRT}
	، مکان‌یابی با استفاده از تگ‌های
	\lr{Aruco}
}


\AUTtitle
%%%%%%%%%%%%%%%%%%%%%%%%%%%%%%%%%%
\vspace*{7cm}
\thispagestyle{empty}
\begin{center}
	\includegraphics[height=5cm,width=12cm]{besm}
\end{center}









%%% -!TEX root = AUTthesis.tex
%% در این فایل، عنوان پایان‌نامه، مشخصات خود، متن تقدیمی‌، ستایش، سپاس‌گزاری و چکیده پایان‌نامه را به فارسی، وارد کنید.
%% توجه داشته باشید که جدول حاوی مشخصات پروژه/پایان‌نامه/رساله و همچنین، مشخصات داخل آن، به طور خودکار، درج می‌شود.
%%%%%%%%%%%%%%%%%%%%%%%%%%%%%%%%%%%%%
%% دانشکده، آموزشکده و یا پژوهشکده  خود را وارد کنید
%\faculty{دانشکده مهندسی برق}
%% گرایش و گروه آموزشی خود را وارد کنید
%\department{گرایش کنترل}
%% عنوان پایان‌نامه را وارد کنید
%\fatitle{بهبود عملکرد ربات اسپایدر و پیاده سازی الگوریتم برنامه ریزی مسیر برای آن
	%	\\[.75 cm]
	%	پایان‌نامه}
%% نام استاد(ان) راهنما را وارد کنید
%\firstsupervisor{دکتر محمداعظم خسروی}
%%\secondsupervisor{استاد راهنمای دوم}
%% نام استاد(دان) مشاور را وارد کنید. چنانچه استاد مشاور ندارید، دستور پایین را غیرفعال کنید.
%%\firstadvisor{نام کامل استاد مشاور}
%%\secondadvisor{استاد مشاور دوم}
%% نام نویسنده را وارد کنید
%\name{سجاد}
%% نام خانوادگی نویسنده را وارد کنید
%\surname{قدیری}
%%%%%%%%%%%%%%%%%%%%%%%%%%%%%%%%%%%
%\thesisdate{مهر 1402}
%
%% چکیده پایان‌نامه را وارد کنید
%\fa-abstract{
	%	ربات‌ها از جدیدترین دستاورد‌های انسان و تلفیقی از فناوری در حوزه‌‌های مختلفی همچون کنترل، هوش مصنوعی و اتوماسیون و... می‌باشند. امروزه هوشمندسازی فناوری های پیشین مورد توجه بوده و ربات‌ها نیز از آن مستثنی نیستند. از میان انواع ربات‌ها آن دسته‌ای که در ساخت آنها از طبیعت الهام گرفته شده است دارای چالش‌های نوینی بوده اند. ربات‌های عنکبوتی چهارپا نمونه‌ای از این دسته هستند. این پژوهش به تلاش هایی که در جهت بهبود عملکرد و مسیریابی ربات عنکبوتی چهارپا انجام شده می‌پردازد. در ابتدا یک ماژول چندجانبه که شامل دوربین می‌باشد به ربات اضافه شده و سپس با استفاده از زبان سی‌ پلاس پلاس به طراحی یک سرور محلی با هدف تبادل اطلاعات بصورت بیسیم بین ربات و آن پرداخته می‌شود. همچنین از این ویژگی برای نظارت و ارزیابی صحت اطلاعات و همچنین تصمیم‌گیری‌های ربات در محیط استفاده خواهد شد. در ادامه تشخیص موانع موجود در محیط به کمک تشخیص رنگ و روش‌های پردازش تصویر با استفاده از زبان پایتون ارائه می‌گردد.
	%	سپس به کمک کتابخانه‌ای که اساس کار آن بهره‌گیری از تگ‌های حاوی اطلاعات می‌باشد، به مکان‌یابی ربات پرداخته می‌شود. سپس شبیه‌‌سازی الگوریتم مسیر‌یابی انجام شده و تغییرات مورد نیاز در جهت تلفیق این الگوریتم با ویژگی‌های ربات برای پیاده‌سازی عملی مطرح می‌گردد.
	%	سپس آزمایش‌های عملی ربات صورت گرفته و نتایج آن بررسی می‌شود.
	%}
%% کلمات کلیدی پایان‌نامه را وارد کنید
%\keywords{ربات عنکبوتی چهارپا ، دوربین، تشخیص موانع، مسیریابی، مکان‌یابی}
%
%
%
%\AUTtitle
%%%%%%%%%%%%%%%%%%%%%%%%%%%%%%%%%%%
%\vspace*{7cm}
%\thispagestyle{empty}
%\begin{center}
%	\includegraphics[height=5cm,width=12cm]{besm}
%\end{center}