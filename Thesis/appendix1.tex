\chapter*{‌پیوست}
\markboth{پیوست}{}
\addcontentsline{toc}{chapter}{پیوست}
%موضوعات مرتبط با متن گزارش پایان نامه كه در يكی از گروه‌های زير قرار می‌گيرد، در بخش پيوست‌ها آورده شوند:
%\begin{enumerate}
	%\item  اثبات های رياضی يا عمليات رياضی طولانی‌.‌
	%\item داده و اطلاعات نمونه (های) مورد مطالعه (\lr{Case Study}) چنانچه طولانی باشد‌.‌
	%\item نتايج كارهای ديگران چنانچه نياز به تفصيل باشد‌.‌
	%\item مجموعه تعاريف متغيرها و پارامترها، چنانچه طولانی بوده و در متن به انجام نرسيده باشد‌.‌
%\end{enumerate}
% براي شماره‌گذاري روابط، جداول و اشكال موجود در پيوست‌ از ساختار متفاوتي نسبت به متن اصلي استفاده مي‌شود كه در زير به‌عنوان نمونه نمايش داده شده‌است. 
% \begin{equation}
%F=ma
%\end{equation}
\section*{کد پایتون \lr{RRT}}
\textbf{ماژول نوشته شده شامل کلاس‌های استفاده شده در فایل اجرایی :}
\begin{latin}
	\lstinputlisting[style=python_style]{code/RRTalgorithm/RRTbasePy.py}
\end{latin}

\textbf{فایل اصلی جهت اجرا :}

\begin{latin}
	\lstinputlisting[style=python_style]{code/RRTalgorithm/RRT.py}
\end{latin}
\section{کد \lr{Localization}}
\textbf{کد مکان‌یابی برای سنجش صحت پارامتر‌های دوربین و ارتباط با ربات}
\begin{latin}
	\lstinputlisting[style=python_style]{code/Localization/Aruco.py}
\end{latin}

