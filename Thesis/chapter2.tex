\chapter{مسیریابی و الگوریتم های پیاده سازی شده}

\section{مقدمه}

در این فصل پایان‌نامه به پردازش تصویر و ارتباط بین ربات و سرور می‌باشد. این فصل به معرفی و بررسی اصول و مبانی پردازش تصویر و نیز نحوه‌ی ارتباط و تبادل اطلاعات بین ربات چهارپا و سرور می‌پردازد. در دنیای امروز که هوش مصنوعی و فناوری‌های رباتیک و در حال پیشرفت های صنعتی می‌باشند، تلفیق پردازش تصویر به عنوان یک افزونه از ربات و ارتباط میان ربات و سرور، برای انجام وظایف پیچیده و کاربردهای متنوع بسیار حائز اهمیت می‌باشد.

این فصل به بررسی مفاهیم پایه پردازش تصویر، تکنیک‌های استخراج و تحلیل اطلاعات از تصاویر، و همچنین نحوه‌ی ارتباط بین ربات و سرور اختصاص دارد. از جمله مسائل مورد بررسی در این فصل، می‌توان به پیش‌پردازش تصاویر، تشخیص الگوها، تصویرسازی داده‌ها و ارسال اطلاعات بین ربات و سرور اشاره کرد.

به عنوان پایه‌ای برای فصل‌های بعدی، این فصل به ما امکان می‌دهد تا اصول پایه پردازش تصویر را برای کنترل و هدایت ربات‌ها بهره‌برداری کنیم. همچنین، اهمیت برقراری ارتباط مؤثر بین ربات و سرور را برای جمع‌آوری داده‌ها و انتقال دستورات کنترلی تاکید می‌کند.

در این فصل، ابتدا به مقدمه‌ای کوتاه درباره موضوع، اهمیت و اهداف این فصل می‌پردازیم. سپس به توضیح دقیق‌تر مفاهیم پایه پردازش تصویر و ارتباطات ربات و سرور می‌پردازیم. در پایان نیز ساختار و مرور مطالب فصل‌های آینده را معرفی خواهیم کرد.

با توجه به اهمیت این فصل در ساختار کلی پایان‌نامه و کاربردهای آینده، مطالب ارائه شده در این فصل بهترین پایه‌ها و مفاهیم را برای فهم بهتر و تحلیل دقیق‌تر موضوعات بحرانی در پژوهش ارائه خواهد داد.
\section{مقدمه‌ای بر پردازش تصویر}
\section{پردازش تصویر}
\subsection{مقدمه}

از مهمترین موضوعات هوش مصنوعی که در سال های اخیر حوزه‌های مختلف به ویژه مهندسی را تحت تاثیرات بسزایی قرار داده‌است، می‌توان به پردازش تصویر و بینایی ماشین اشاره کرد. کاربردهای کنترلی همچون ماشین‌های خودران، دوربین‌های کنترل جرایم رانندگی، سیستم‌های تشخیص چهره و... از پردازش تصویر بهره می‌برند.
\subsection{تصویر چیست؟}

تصویر، نوعی نمایش بصری از داده‌هاست که معمولاً توسط دوربین‌ها یا سنسورهای تصویربرداری ثبت می‌شود.تصویر به صورت مجموعه‌ای از نقاط کوچک به نام پیکسل‌ها تشکیل می‌شود که هر کدام حاوی اطلاعاتی چون رنگ و روشنایی می‌باشند. از طریق ترکیب مختلف پیکسل‌ها، تصویرهایی با پیچیدگی، اشکال و جزئیات مختلف ایجاد می‌شوند.

تصاویر می‌توانند در انواع مختلف از جمله تصاویر رنگی، تصاویر سیاه و سفید و... باشند. از تصاویر به عنوان یک ابزار قوی برای انتقال اطلاعات و نمایش داده‌ها در زمینه‌های مختلف از جمله در پزشکی، هنر، رباتیک و...استفاده می‌شود.
\subsection{پردازش تصویر چیست؟}

پردازش تصویر یک زیرشاخه از پردازش سیگنال است که به تحلیل، تغییر، و بهبود تصاویر دیجیتالی می‌پردازد. در این فرآیند، تصاویر از دوربین‌ها یا سنسورهای تصویربرداری گرفته می‌شوند و سپس توسط الگوریتم‌ها و روش‌های پردازشی مختلف تجزیه و تحلیل می‌شوند.

هدف اصلی پردازش تصویر بهینه‌سازی و تحسین کیفیت تصاویر است. این می‌تواند شامل کاهش نویز، تعدیل رنگ و کنتراست، تشخیص الگوها، تشخیص ویژگی‌ها و استخراج اطلاعات از تصاویر باشد. در زمینه‌های مختلف از پردازش تصویر استفاده می‌شود که از جمله آنها می‌توان به پزشکی (تشخیص بیماری‌ها از تصاویر پرتونگاری)، رباتیک (شناسایی محیط توسط ربات‌ها)، و امنیت (تشخیص چهره‌ها و اجسام در تصاویر نظارتی) اشاره کرد.

پردازش تصویر شامل مراحل مختلفی مانند پیش‌پردازش، تبدیلات، تشخیص الگوها، و استخراج ویژگی‌ها است. الگوریتم‌های پردازش تصویر معمولاً بر اساس مفاهیم ریاضی، آمار و هوش مصنوعی طراحی می‌شوند. از این رو می‌توان گفت پردازش تصویر در حوزه‌های مختلف از مهندسی، علوم رایانه، و علوم پایه کاربردهای متعددی داشته و از اهمیت بالایی در تحلیل و انتقال اطلاعات تصویری برخوردار است.

\section{الگوریتم درخت جستجوی تصادفی}
\subsection{مقدمه}
همانطور که در بخش
\ref{مروری بر الگوریتم‌های مسیریابی}
اشاره شد، الگوریتم‌های متنوعی در رباتیک کاربرد دارد.
در حوزه‌ی رباتیک، پیشرفت‌های قابل توجهی در الگوریتم‌های برنامه‌ریزی مسیر رخ داده است که به ربات‌های خودکار امکان می‌دهد تا به طور کارآمد در محیط‌های پیچیده حرکت کنند. از میان این الگوریتم‌ها، الگوریتم 
\textbf{درخت جستجوی تصادفی}
\noindent\unskip\LTRfootnote{Rapidly Exploring Random Tree}
یا به اختصار
\lr{RRT}
به عنوان یک رویکرد نوآورانه شناخته می‌شود که به حل چالش یافتن مسیرهای قابل اجرا در فضاهای تنظیم با بعد بالا می‌پردازد. 

این الگوریتم نیز با استفاده از جستجوی تصادفی فضای پیکربندی به حل برنامه مساله‌ریزی مسیر می‌پردازد. در این روش، ریشه درخت در پیکربندی اولیه قرار دارد. ابتدا یک نقطه به صورت تصادفی با توزیع یکنواخت
\noindent\unskip\LTRfootnote{Uniform distribution}
انتخاب می‌شود. اگر نقطه انتخاب شده به فضای موانع تعلق داشته باشد، نقطه دیگری انتخاب می‌شود. با انتخاب نقطه از فضای آزاد، در راستای خط واصل بین این نقطه و ریشه درخت، نقطه جدیدی تحت عنوان
\lr{X\_new}
که به اندازه گام ربات از ریشه فاصله دارد، انتخاب می‌شود.
اگر بتوان
\lr{X\_new}
را با پاره‌خطی که تماما به فضای آزاد تعلق دارد، به ریشه وصل کرد،
\lr{X\_new}
به عنوان گره جدید به درخت اضافه می‌شود.
نمایی از درخت توسعه یافته با این روش در شکل
\ref{RRT}
نمایش داده شده است.
\begin{figure}[H]
	\centering
	\includegraphics[width=0.6\textwidth]{./images/Chapter2/RRT}	
	\caption[الگوریتم \lr{RRT}]{الگوریتم \lr{RRT} \cite{MohadesKhorasani}}
	\label{RRT}
\end{figure}
\noindent
\unskip
%این الگوریتم به عنوان یک روش مبتنی بر هیوریستیک
% \noindent\unskip\LTRfootnote{Heuristic}
%، با استفاده از تصادف و کاوش به سرعت یک ساختار مانند درخت را ایجاد می‌کند که این امر آن را برای کاربردهای بهینه‌سازی زمان و محیط‌های پویا مناسب می‌کند. اصول اساسی این الگوریتم در قابلیت اکتشاف مؤثر فضاهای تنظیمی و تنوع آن، آن را به یک اصول اصلی در تحقیقات و کاربردهای رباتیک مدرن تبدیل کرده است.

%الگوریتم
%\lr{RRT}
%که در حوزه‌ی برنامه‌ریزی حرکت معرفی شده است، یک راه حل نوین برای چالش پیچیده یافتن مسیر برای عوامل خودکار در فضاهای با بعد بالا ارائه می‌دهد. در اصل، این روش به دنبال برقراری تعادل بین کاوش و بهره‌برداری می‌گردد. این الگوریتم با یک تنظیم اولیه آغاز می‌شود و به صورت تکراری با نمونه‌برداری تصادفی از نقاط در فضای تنظیمی، آن‌ها را به نزدیک‌ترین نقطه در درخت موجود متصل می‌کند. این رویکرد تضمین می‌کند که الگوریتم مناطقی از فضا را که هنوز کاوش نشده‌اند، کاوش کند، در عین حال به طور تدریجی از مناطقی که قبلاً کاوش شده‌اند بهره‌برداری می‌کند.

%یکی از ویژگی‌های برجسته‌ی
%\lr{RRT}
%توانایی تطبیق به محیط‌ها و محدودیت‌های مختلف است. با تشکیل تک تک نقاط درخت به طور تدریجی، رشد الگوریتم به سمت مناطقی با کاوش کمتر سوق می‌شود، که این ویژگی آن را به ویژه برای موقعیت‌هایی که فضای تنظیم مشخص نشده یا با موانع پراکنده است، مناسب می‌کند. علاوه بر این،
%\lr{RRT}
%همگام‌سازی سریع را اجرا می‌کند که به آن اجازه می‌دهد که مسیرهای قابل اجرا را به سرعت ایجاد کند، حتی در محیط‌های پیچیده و پویایی.

%قابلیت‌های این الگوریتم در کاربردهای زمان‌واقع واقعی نیز یک جنبه جذاب دیگر است. به دلیل ساختار تکراری و تطبیق‌پذیری، این الگوریتم برای مواردی که تصمیم‌گیری باید سریع و واکنش‌پذیر باشد، مناسب است. علاوه بر این، محققان اقدام به توسعه
%\lr{RRT}
%برای مواجهه با چالش‌های خاص، مانند موانع پویا و سیستم‌های چندعاملی و... کرده‌اند که بیشتر نشان می‌دهد چقدر این الگوریتم چندمنظوره و عملی است.

%به عبارتی، الگوریتم
%\lr{RRT}
%یک الگوریتم نوآورانه برای برنامه‌ریزی مسیرهای خودکار است. با ترکیب کاوش و بهره‌برداری ،
%\lr{RRT}
%به طور موثر در فضاهای تنظیمی با بعد بالا حرکت کرده و به شکل مناسبی به شرایط محیطی مختلف تطبیق می‌یابد. چندمنظورگی، قابلیت‌های زمان‌واقع و تطبیق‌پذیری با چالش‌های خاص،
%\lr{RRT}
%را به یک ابزار ارزنده در تحقیقات و کاربردهای رباتیک مدرن تبدیل کرده است و پیشرفت‌هایی در خودکاری، کارایی و ایمنی ایجاد می‌کند.


\subsection{شبیه‌سازی دوبعدی با پایتون}
\subsubsection{نتایج شبیه‌سازی}
در ابتدا نتایج حاصل از شبیه‌سازی توسط پایتون قرار داده‌ شده‌است. همانطور که در شکل
\ref{نتیجه شبیه‌سازی RRT}
قابل مشاهده است، یک محیط با ابعاد مشخص و تعداد موانع مشخص طراحی و سپس با تعیین نقطه شروع و پایان برای ربات، با استفاده از الگوریتم پیشنهادی ربات توانسته است مسیر خود را بدون برخورد با موانع بیابد.
\begin{figure}[H]
	\centering
	\includegraphics[width=0.7\textwidth]{./images/Chapter2/ConvexResult}	
	\caption[شبیه سازی الگوریتم \lr{RRT}]{شبیه سازی الگوریتم \lr{RRT}}
	\label{نتیجه شبیه‌سازی RRT}
\end{figure}
\noindent
\unskip

\subsubsection{شبیه‌کد و تشریح جزئیات برنامه}
در زیر شبه‌کد این بخش مختصرا قراره داده شده و پس از آن به تشریح کامل‌تر آن پرداخته خواهد شد.
\section*{}
\begin{latin}
	\lstinputlisting[style=python_style]{code/Pseudocodes/RRTPseudocode.py}
\end{latin}

همانطور که در شبه‌کد بالا مشخص است برنامه تا زمانی که آخرین گره گراف به ناحیه نزدیک نقطه هدف نرسد ادامه خواهد داشت. جزئیات و نحوه تولید مسیر در قطعه برنامه‌ای در پیوست موجود می‌باشد. نحوه عملکرد توابع اصلی برنامه برای تولید مسیر به شرح زیر است:
\begin{enumerate}
	\item
	تولید یک گره تصادفی با فاصله مشخص از گره شروع
	\item
	بررسی معتبر بودن گره جدید (معتبر بودن به معنای آنکه آیا گره جدید بر روی موقعیت یکی از موانع نبوده باشد.)
	\item
	در صورت معتبر بودن گره جدید با فواصل از پیش تعیین شده توسط چند قدم یال‌هایی برای گراف ایجاد می‌شود.
	\item
	اگر در طی این چند قدم یکی از یال‌ها با یکی از موانع برخورد داشته باشد، قدم دیگری برای رسیدن به گره جدید انتخاب می‌شود. در غیر این صورت برنامه تا رسیدن به گره جدید ادامه داده و دوباره به مرحله اول(تولید گره تصادفی) برگشت داده خواهد شد.
	\item
	گراف این روند را تا زمانی که به نقطه هدف نزدیک نشود ادامه داده و در نهایت با توجه به دانستن موقعیت این نقطه، مسیر بهترین مسیر از بین مسیر‌های ساخته شده توسط گره‌های خود را انتخاب می‌کند.
\end{enumerate}

\begin{figure}[H]
	\centering
	\includegraphics[width=0.8\textwidth]{./images/Chapter2/NewNodeSteps_without_background}	
	\caption[مراحل تولید گره جدید و تشکیل مسیر]{مراحل تولید گره جدید و تشکیل مسیر\cite{Algobotics}}
	\label{مراحل تولید گره جدید}
\end{figure}
\noindent
\unskip

برای درک بهتر مراحل بالا می‌توان شکل
\ref{مراحل تولید گره جدید}
را در نظر گرفت. همچنین قابل ذکر است که برای هر یک از مراحل بالا یک یا دو متد
\noindent\unskip\LTRfootnote{Method}
در ماژول نوشته شده در نظر گرفته شده است و این متد‌ها با یکدیگر در ارتباط بوده و یا به صورت تو در تو
 \noindent\unskip\LTRfootnote{Nested}
در یکدیگر استفاده شده‌اند. 

\subsubsection{ادغام الگوریتم با قابلیت‌های ربات به منظور پیاده‌سازی}
همانطور که مشخص است استفاده از الگوریتم پیشنهادی به تنهایی مفید واقع نخواهد شد و بایستی از قابلیت های موجود در ربات برای بهینه‌سازی این روش استفاده شود. از قابلیت های ربات که می‌تواند به کار آید، دوربین استفاده شده می‌باشد. با استفاده از دوربین موجود موانع و فاصله‌ی ربات تا موانع تشخیص داده شده و سپس با توجه به این اطلاعات ربات تصمیم می‌گیرد تا نقاط تصادفی خود را در کدام جهت تولید کند و یا در کدام جهت به تولید ادامه مسیر نپردازد. بدین منظور در بخش های آتی به تشریح موارد ذکر شده پرداخته خواهد شد.

\newpage
\section{\lr{Localization}}

\subsection{مقدمه}

مکان‌یابی یکی از جنبه‌های حیاتی در علم رباتیک است که به ربات‌ها امکان می‌دهد تا مکان و موقعیت دقیق خود را در محیط تعیین کنند. برای ربات‌ها، داشتن اطلاعات صحیح و دقیق در مورد مکان خود بسیار مهم است تا بتوانند به درستی عمل کرده و وظایف مختلفی را انجام دهند. در این بخش به مکان‌یابی در ربات عنکبوتی چهارپا پرداخته خواهد‌شد. بدین منظور به انتخاب یکی از روش‌‌‌های نوین در این حوزه دست زده شد. در سال های اخیر برخی از آزمایشگاه های رباتیک در دانشگاه‌‌‌های معتبر جهان که از میان آنها می‌توان به آزمایشگاه رباتیک اپریل
\noindent\unskip\LTRfootnote{APRIL robotics lab}
در دانشگاه میشیگان اشاره کرد، روش‌هایی برای مکان‌یابی با استفاده از تگ‌هایی مشابه به تگ‌های معروف که با نام 
\textbf{کیوآر کد}
\noindent\unskip\LTRfootnote{QR code}
شناخته شده‌اند، معرفی کرده اند.

\begin{figure}[H]
	\centering
	\includegraphics[width=0.8\textwidth]{./images/Chapter2/AprilTagLab}	
	\caption[آزمایشگاه اپریل میشیگان]{ آزمایشگاه اپریل میشیگان\cite{Apriltag}}
	\label{AprilTagLab}
\end{figure}
\noindent
\unskip

\newpage
\subsection{استفاده از تگ}
\subsubsection{مفاهیم اصلی}
نحوه استفاده از تگ‌های این چنینی بر اساس روابط مثلثاتی و هندسه اقلیدسی موجود در فضا است و با داشتن اطلاعاتی همچون فاصله کانونی دوربین و مرکز آن، اطلاعاتی همچون فاصله در هر سه محور و یا زاویه دوربین نسبت به تگ را برمی‌گردانند.
بدین منظور از یکی از ماژول‌های موجود در کتابخانه‌ی
\lr{OpenCV}
که
\lr{Aruco}
نام دارد، استفاده خواهد شد. 
\subsubsection{شبه‌کد و تشریح جزئیات برنامه}

در زیر شبه‌کد این بخش مختصرا قراره داده شده و پس از آن به تشریح کامل‌تر آن پرداخته خواهد شد.
\section*{}
\begin{latin}
	\lstinputlisting[style=python_style]{code/Pseudocodes/ArucoPseudocode.py}
\end{latin}

مکان‌یابی در رباتیک به وسیله مختصات مکانی (مثلاً نقاط
\lr{(x, y, z)}
یا مختصات خروجی (طول، عرض، ارتفاع) صورت می‌پذیرد. این اطلاعات توسط تگ‌های
\lr{Aruco}
به ربات ارائه می‌شود. این تگ‌ها اطلاعاتی را شامل می‌شوند برای مثال سه زاویه
\lr{(roll, pitch, yaw)}
برای جهت‌یابی و فاصله‌ها در سه محور مختلف می‌شوند. برای دقت بیشتر در محاسبات، ما یک ماتریس را تعریف می‌کنیم که حاوی فاصله‌های مرکزی دوربین و فوکوس آن است و این ماتریس را به کد مکان‌یابی ارائه می‌دهیم. این اطلاعات به ربات امکان می‌دهند تا موقعیت و جهت خود را با دقت بیشتری تعیین کنند و وظایف خود را انجام دهند.
\section*{}
\begin{latin}
	\lstinputlisting[style=python_style]{code/Localization/Aruco.py}
\end{latin}



\section{جمع بندی}
در این فصل با تشریح مفاهیمی از تصویر و پردازش آن و کاربرد‌های آن در زمینه‌های مختلف همچون رباتیک آغاز و با بررسی الگوریتم درخت جستجوی تصادفی و شبیه‌سازی آن در محیط پایتون ادامه یافت. همچنین در انتها به بررسی مفهومی بسیار مهم به نام مکان‌یابی که برای پیاده‌سازی عملی الگوریتم بر روی ربات در محیط آزمایشگاهی ضروری است اشاره و از راهکاری نوین که در سال‌های اخیر مورد توجه قرار گرفته است، استفاده شد.
